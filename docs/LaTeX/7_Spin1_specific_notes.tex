\section{Spin1 specific notes}

\begin{enumerate}
\item This spin has several modifications. Start these modifications by cut the wrong traces, and adding jumpers as shown below.

\item You have the PCB + components, so get KICAD and the files, or the pdf files from puma.freeems.org. PDFs aren\'t search-able so you may want to choose to install KICAD until we workout that.

\item Start the assembly!

Just solder the components. If you ordered a board, you should know how to do it. An oven is a fast way to get it done.

Don\'t put too much paste for the small regulator, or it will get misaligned.

Components that shouldn\'t be populated:

\begin{itemize}
\item F1, F3 (Fuses)
\item R226, R227, Q18, Q19, bridge pin 1 and 3 of Q19 (this is the shutdown circuit)
\item R133 (bad pullup)
\item R228 OR R229, using one of them defines whether the XOR negates or not its outputs.
\item If you use VR inputs, R212, R213, R215, and R216 should be bigger, like ¼ or ½W. 10Kohm to 20 kohm will be fine.
\item U18, R186, R187, C107, D74, D75, C106 (thermocouple driver)
\end{itemize}

\item Program the MCU using a BDM pod.

Install Codewarrior, open the programmer, go to File $\rightarrow$ Load application, and select the .s12 (FreeEMS serial monitor).

It should get connected, program it, verify, and never complain.

\item Load FreeEMS firmware, using seank\'s loader.

\item Install  MTX and connect to the board to the PC to check that freeems is running.

\end{enumerate}
